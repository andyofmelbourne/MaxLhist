\documentclass[11pt]{article}
\usepackage{amsmath}
%Gummi|065|=)
\title{\textbf{Finding adu distributions or finding photons}}
\author{Andrew Morgan}
\date{}
\usepackage{graphicx}
\begin{document}

\maketitle

\section{Maximum Likelihood for a single random variable}
Thought experiment:

\begin{itemize}

  \item We have a single discrete random variable x with a distribution $f(x)$
  \item $x$ is measured $N$ times but with an offset $f(x - \mu)$
  \item This is repeated $M$ times with $M$ different offsets
  \item $h^m_i$ is the histogram of $x_m$, the values of x on the m'th pixel, for $x_m = i$ in $0 \rightarrow I-1$

\end{itemize}


The probability of measuring our sequence of values $ x^m_n $ is:
\begin{align}
   Pr(x^m_n; f) &= \Pi_m^M \Pi_n^N f(x^m_n - \mu_m) = \Pi_m^M \Pi_i^I f(i - \mu_m)^{h^m_i}
\end{align}

The log likelihood is:
\begin{align}
   \ln(Pr(x^m_n; f)) &= \sum_m^M \sum_i^I h^m_i \ln(f(i - \mu_m))
\end{align}

Our log likelihood error is:
\begin{align}
   \varepsilon(\mu_n, f_i) &= -\sum_m^M \sum_i^I h^m_i \ln(f(i - \mu_m))
\end{align}
with a minimum value of 0. 


\begin{verbatim}

def log_likelihood_calc(f, mus, hists, prob_tol = 0.0):
    """
    Calculate the log likelihood error given a probability distribution f,
    a set of shifts mus, the measured histograms for each shift hists.
    
    log likelihood error = - sum_m sum_I hists[m, I] * ln( f[i-mus[m]] )
        
    As the pixel shifts need not be integer f[i-mus[m]] is calculated
    using the Fourier shift theorem and uses the function roll_real.
    
    Parameters
    ----------
    f : float array
        Real space values of f of length I.
    mus : float array
        The value in pixels of the shift amount of length M.
    hists : 2 dimensional integer array
        The measured values sampled from f shifted by the mus.
        hists must have the shape (M, I).
    prob_tol : float, optional
        Amount to add to f to aviod -infinity in the natural logarithm.
            
    Returns
    -------
    log_likelihood_error : float
        The log likelihood error.
    """
    error = 0.0
    for m in range(len(hists)):
        # only look at adu or pixel values that were detected on this pixel
        Is = np.where(hists[m] > 0)
        
        # evaluate the shifted probability function
        fs = roll_real(f, mus[m])[Is] 

        # sum the log liklihood errors for this pixel
        e  = hists[m, Is] * np.log(prob_tol + fs)
        error += np.sum(e)
    return -error


\end{verbatim}

\subsection{$\mu$ The offset's or dark values}
We could demand that the $\mu$'s are integer, so that shifting $f(i)$ is well defined. But this leads to problems in the minimization, since most refinement algorithms like a continuous set of variables and a continuous error function. Also we would expect that the shift of the distribution (or ``dark value") at a pixel may not be an integer multiple of counts. 

Another issue is that adding a constant to the mu's also corresponds to an overall shift in the probability distributions: $\mu + c \longleftrightarrow  f(i - c)$. We can remove this redundancy by demanding that the sum of the mu's is equal to 0. This can be automatically satisfied if we express the independent variables in Fourier space. Moving to Fourier space we have:

\begin{align} 
   \mu_m &= \frac{1}{M} \sum_{l=0}^{M-1} \hat{\mu}_l e^{2\pi i \frac{m l}{M}}, \\
   &= \frac{1}{M} \sum_{f_l=-M/2 + 1}^{M/2} \hat{\mu}_{f_l} e^{2\pi i mw_l} 
\end{align}

Since $\mu$ is real the reciprocal space representation has Hermitian symmetry. That is, the negative Fourier frequencies are the complex conjugate of the positive Fourier frequencies. In the discrete Fourier transform one can interpret the first element as the zero frequency of $\mu$(which is useful because we can demand that this is zero), the first half of the array as the positive frequencies and the second half of the array as the negative frequencies. Because of this it is useful switch between the array index, $m$ in this case, and the Fourier index, $f_l$. Thankfully Python already has a function for doing this:

\begin{verbatim}
 l   0  1  2  3  4  5  6  7    range(M)
 fl  0  1  2  3  4 -3 -2 -1    np.fft.fftfreq(M, d=1/float(M))
 wl (0  1  2  3  4 -3 -2 -1)/M np.fft.fftfreq(M)
\end{verbatim}

So now, our independent values for $\mu$ are:
\begin{align}
   \hat{\mu}_l \text{  for } l &= 1, 2, 3 .. M/2 + 1 \text{  where}\\
   \hat{\mu}_0 &= 0 \text{  and} \\
   \hat{\mu}_{l_f} &= \hat{\mu}^*_{-l_f}
\end{align}

\newpage
Let's make a function that returns the real-space real function corresponding to the Fourier components givin by Eq. (6).
\begin{verbatim}

def make_f_real(fhats, f_norm = 0.0):
    """
    Compute the real space representation of f given that 
    f is real and that sum_i f_i = f_norm. The fhats are 
    the complex Fourier space values of f from 1 to 
    N // 2 + 1 where N is the size of the real space array.
    
    Parameters
    ----------
    fhats : 1d complex array
        Fourier space values of f.
    f_norm : float, optional
        Sum of the real space function.
              
    Returns
    -------
    f : 1d float array
        the realspace representation of fhats given f_norm.
    """
    fh = np.concatenate( ([f_norm], fhats) )
    f  = np.fft.irfft(fh)
    return f
    
\end{verbatim}

The probability function f must also be normalized, but to 1. So it to should expressed in Fourier space. Now we must define how the shift will happen. Using the Fourier shift theorem we have:

\begin{align}
f_{i - \mu_m} = \frac{1}{I} \sum_{j=0}^{I-1} (\hat{f}_j e^{-2\pi i \mu_m w_j}) e^{2\pi i i w_j}
\end{align}

\begin{verbatim}

def shift_f_real(f, shift):
    """
    Apply the Fourier shift algorithm to f by shift pixels.
    
    Parameters
    ----------
    f : float array
        Real space values of f.
    shift : float
        The value in pixels of the shift amount.
            
    Returns
    -------
    f_shift : 1d float array
        The shifted representation of f.
    """
    fh      = np.fft.rfft(f)
    ramp    = np.exp(-2.0J * np.pi * shift * np.arange(float(fh.shape[0])) \
                     / float(f.shape[0]))
    f_shift = np.fft.irfft(fh * ramp)
    return f_shift

\end{verbatim}

Now we know have to define the independent variables for the probability (or adu) distribution $f$, the shift variables (or dark values) $\mu$. We can also calculate the real space values and shift the probability function. 

If we want to refine the set of $\mu$ values using a gradient based method then we must calculate the gradients. We need to find the vector gradient of the log likelihood error with respect to the independent $\mu$-variables. 

This is going to get messy... Let us try to evaluate the derivative with respect to the real part of the positive Fourier components of $\mu$ ($\hat{\mu}^r$). First we go down the derivative rabbit hole, then we clime our way back up: 

\begin{align}
   \frac{\partial \varepsilon(\mu, f)}{\partial \hat{\mu}^r_n} &= \frac{\partial}{\partial \hat{\mu}^r_n}\left[-\sum_m^M \sum_i^I h^m_i \ln(f(i - \mu_m))\right] \\
   &=  -\sum_m^M \sum_i^I h^m_i \frac{\partial}{\partial \hat{\mu}^r_n} \ln(f(i - \mu_m)) \\
   &=  -\sum_m^M \sum_i^I \frac{h^m_i}{f(i - \mu_m)} \frac{\partial f(i - \mu_m)}{\partial \hat{\mu}^r_n}
\end{align}


Fine, now need to evaluate the final term:
\begin{align}
   \frac{\partial f(i - \mu_m)}{\partial \hat{\mu}^r_n} &= \frac{\partial}{\partial \hat{\mu}^r_n} \frac{1}{I} \sum_{j=0}^{I-1} (\hat{f}_j e^{-2\pi i \mu_m w_j}) e^{2\pi i i w_j} \\
   &= \frac{1}{I} \sum_{j=0}^{I-1} (\hat{f}_j e^{-2\pi i \mu_m w_j}) e^{2\pi i i w_j} \left[-2\pi i w_j \frac{\partial \mu_m}{\partial \hat{\mu}^r_n} \right]
\end{align}


OK, and the last term again:
\begin{align}
   \frac{\partial \mu_m}{\partial \hat{\mu}^r_n} &= \frac{\partial}{\partial \hat{\mu}^r_n} \left[\frac{1}{M} \sum_{f_l=-M/2 + 1}^{M/2} \hat{\mu}^r e^{2\pi i mw_l} \right] \\ 
   &= \frac{1}{M} \sum_{f_l=-M/2 + 1}^{M/2} \frac{\partial \hat{\mu}^r_{f_l}}{\partial \hat{\mu}^r_n} e^{2\pi i mw_l} 
\end{align}


We have to be carefull here. Remember that $\hat{\mu}$ is Hermitian, so not all of the $\hat{\mu}^r$ are independent:
\begin{align}
   \frac{\partial \mu_m}{\partial \hat{\mu}^r_n} &= \frac{1}{M} \sum_{f_l=-M/2 + 1}^{-1} \delta_{f_n + f_l} e^{2\pi i mw_l} + \frac{1}{M} \sum_{f_l=1}^{M/2-1} \delta_{f_n - f_l} e^{2\pi i mw_l} \\
   &+ \frac{1}{M} \delta_{f_n - 0} + \frac{1}{M} \delta_{f_n - M/2} e^{\pi i m} \\
   &= \frac{1}{M} \sum_{f_l=1}^{M/2-1} \delta_{f_n - f_l} \left[e^{-2\pi i mw_n} + e^{2\pi i mw_n} \right] + \frac{1}{M} \delta_{f_n - 0} + \frac{1}{M} \delta_{f_n - M/2} (-1)^m\\
   &= \frac{2}{M} \cos(2\pi mw_n) \sum_{f_l=1}^{M/2-1} \delta_{f_n - f_l} + \frac{1}{M}\left(\delta_{f_n - 0} +  (-1)^m \delta_{f_n - M/2} \right)
\end{align}

Notice those two extra anoying bits on the right? Well we don't have to worry about $n = 0$ because are keeping the $\mu$ values normalised to 0, but that $(-1)^m$ comes about when the array dimensions are even. Hmmm... I guess we should keep M even for now but I am sure it cannot be that hard to work out the odd case. 

\pagebreak
While we are here I guess we should look at the derivative with respect to $\hat{\mu}^i$:
\begin{align}
   \frac{\partial \mu_m}{\partial \hat{\mu}^i_n} &= -\frac{i}{M} \sum_{f_l=-M/2 + 1}^{-1} \delta_{f_n + f_l} e^{2\pi i mw_l} + \frac{i}{M} \sum_{f_l=1}^{M/2-1} \delta_{f_n - f_l} e^{2\pi i mw_l} \\
   &= \frac{1}{M} \sum_{f_l=1}^{M/2-1} \delta_{f_n - f_l} \left[-ie^{-2\pi i mw_n} + ie^{2\pi i mw_n} \right] \\
   &= -\frac{2}{M} \sin(2\pi mw_n) \sum_{f_l=1}^{M/2-1} \delta_{f_n - f_l} 
\end{align}

We can bring the real and imaginary parts together:
\begin{align}
   \frac{\partial \mu_m}{\partial \hat{\mu}_n} &\equiv \frac{\partial \mu_m}{\partial \hat{\mu}^r_n} + i \frac{\partial \mu_m}{\partial \hat{\mu}^i_n} \\
   &= \frac{2}{M} \left[\cos(2\pi mw_n) - i\sin(2\pi mw_n)\right] \sum_{f_l=1}^{M/2-1} \delta_{f_n - f_l} \\
   &+ \frac{1}{M}\left(\delta_{f_n - 0} +  (-1)^m \delta_{f_n - M/2} \right) \\
   &= \frac{2}{M} e^{-2\pi i m w_n} \sum_{f_l=1}^{M/2-1} \delta_{f_n - f_l} + \frac{1}{M}\left(\delta_{f_n - 0} +  (-1)^m \delta_{f_n - M/2} \right)
\end{align}

As an aside, one can view the two dimensional matrix in Eq. (27) as a discrete cosine transformation matrix of the first kind http://docs.scipy.org/doc/scipy-0.15.1/reference/generated/scipy.fftpack.dct.html (after multiplying by M).

At this point we are at the bottom of the rabitt hole. And remember that it is easier to go down than up! 

Put Eq. (27) back into Eq. (14):
\begin{align}
   \frac{\partial f(i - \mu_m)}{\partial \hat{\mu}_n} &= \frac{1}{I} \sum_{j=0}^{I-1} (\hat{f}_j e^{-2\pi i \mu_m w_j}) e^{2\pi i i w_j} \left[-2\pi i w_j \frac{\partial \mu_m}{\partial \hat{\mu}_n} \right] \\
   &= \frac{\partial \mu_m}{\partial \hat{\mu}_n} \frac{1}{I} \sum_{j=0}^{I-1} (-2\pi i w_j \hat{f}_j e^{-2\pi i \mu_m w_j}) e^{2\pi i i w_j} 
\end{align}


There are three essensial parts to Eq. (22) first is the derivative term $\frac{\partial \mu_m}{\partial \hat{\mu}_n}$ which is a two dimensional function outside of the sum. The second is the multimplication (inside the sum) by $e^{-2\pi i \mu_m w_j}$ which as we already know has the effect of shifting $f$ in real space by $\mu_m$. The final part is the multiplication by $-2\pi i w_j$, in real space this is like taking the spatial derivative of $f$. As the second and third parts commute (they are both multiplications in Fourier space) I will define their effect with this notation:
\begin{align}
   f'_i &\equiv \frac{1}{I} \sum_{j=0}^{I-1} -2\pi i w_j \hat{f}_j e^{2\pi i i w_j} \\
   f'_{i-\mu_m} &\equiv \frac{1}{I} \sum_{j=0}^{I-1} (-2\pi i w_j \hat{f}_j e^{-2\pi i \mu_m w_j}) e^{2\pi i i w_j} 
\end{align}

\begin{verbatim}

def grad_shift_f_real(f, shift):
    """
    Apply the Fourier shift algorithm to f by shift pixels. 
    In addition take the gradient of f.
    
    Parameters
    ----------
    f : float array
        Real space values of f.
    shift : float
        The value in pixels of the shift amount.
            
    Returns
    -------
    f_shift : 1d float array
        The gradient of the shifted representation of f.
    """
    fh      = np.fft.rfft(f)
    lramp   = -2.0J * np.pi * np.arange(float(fh.shape[0])) \
              / float(f.shape[0])
    ramp    = lramp * np.exp(shift * lramp)
    f_shift = np.fft.irfft(fh * ramp)
    return f_shift

\end{verbatim}

\iffalse
\begin{figure}[htp]
\centering
\includegraphics[scale=0.50]{/home/amorgan/Physics/git_repos/MaxLhist/shifted_gradient.png}
\caption{Example of the shifted gradient of a gaussian. Using grad\_shift\_f\_real.}
\label{shifted_grad}
\end{figure}
\fi

With that done, I will use the notation from Eq. (31) to express the derivative of the probability function with respect to the real and imaginary components of the Fourier spectrum of the shift values:
\begin{align}
   \frac{\partial f(i - \mu_m)}{\partial \hat{\mu}_n} &=  \frac{\partial \mu_m}{\partial \hat{\mu}_n} \times f'_{i-\mu_m}
\end{align}


Now we can go back to Eq. (12):
\begin{align}
   \frac{\partial \varepsilon(\mu, f)}{\partial \hat{\mu}_n} &= -\sum_m^M \sum_i^I \frac{h^m_i}{f(i - \mu_m)} \frac{\partial \mu_m}{\partial \hat{\mu}_n} \times f'_{i-\mu_m} \\
   &= -\sum_m^M  \frac{\partial \mu_m}{\partial \hat{\mu}_n} \sum_i^I \frac{h^m_i}{f(i - \mu_m)} \times f'_{i-\mu_m}
\end{align}


As it stands we have function of $i$ (the adu values) and $m$ (the pixel numbers) just before the sum over the adu values. Then we have a transform from $m\rightarrow n$, where the transform matrix is given by $\frac{\partial \mu_m}{\partial \hat{\mu}_n}$ which is independant of the model and the data. So what does this transform look like?
\begin{align}
   g_n &= \sum_{m=0}^{M-1} \left[ \frac{2}{M} e^{-2\pi i m w_n} \sum_{f_l=1}^{M/2-1} \delta_{f_n - f_l} + \frac{1}{M}\left(\delta_{f_n - 0} +  (-1)^m \delta_{f_n - M/2} \right) \right] \times h_m
\end{align}

This is rather strange, note that the transform sums over $m$ not $n$, so it does not reduce to the discrete cosine transform of the first kind. Regardless, 
\begin{align}
   g_n &= \frac{1}{M} \sum_{m=0}^{M-1} h_m                      &&\text{for } n = 0 \\
   &= \frac{2}{M} \sum_{m=0}^{M-1} e^{-2\pi i m w_n} \times h_m &&\text{for } 0 < n < \frac{M}{2} \\
   &= \frac{1}{M} \sum_{m=0}^{M-1} (-1)^m h_m                   &&\text{for } n = \frac{M}{2}
\end{align}

\pagebreak
\begin{verbatim}

def mu_transform(h):
    """
    Calculate a strange transform. So far only works
    when h.shape[0] is an even number (and h is 1D).
    Note that in the output n = 0 is excluded.
	
    g[n] = 1/M sum_m=0^M-1 h[m]                         for n = 0
    g[n] = 2/M sum_m=0^M-1 e^(-2 pi i m n / M) h[m]     for 0 < n < M/2
    g[n] = 1/M sum_m=0^M-1 (-1)^m h[m]                  for n = M/2
    
    Parameters
    ----------
    h : 1D float array
        Values of h of length M.
            
    Returns
    -------
    g : 1D complex array
        g for n = 1, ..., M/2, so with the shape M/2 
    """
    if h.shape[0] % 2 == 1 :
        raise ValueError('input array shape must even for now')
    g = np.zeros((h.shape[0] / 2, ), dtype = np.complex128)
    
    #g[0]     = np.sum(h) / float(M)
    g[1 : -1] = np.fft.fft(h)[1 : M/2] * 2. / float(M)
    g[-1]     = np.sum(h * (-1)**np.arange(h.shape[0]) ) / float(M)
    return g
\end{verbatim}

\end{document}
